% Atur variabel berikut sesuai namanya

% nama
\newcommand{\name}{M. Dafa Raisya Rajwa}
\newcommand{\authorname}{Rajwa, M. Dafa Raisya}
\newcommand{\nickname}{Dafa}
\newcommand{\advisor}{Ahmad Zaini, S.T., M.Sc.}
\newcommand{\coadvisor}{Dr. Eko Mulyanto Yuniarno, S.T., M.T.}
\newcommand{\examinerone}{Penguji I}
\newcommand{\examinertwo}{Penguji II}
\newcommand{\examinerthree}{Penguji III}
\newcommand{\headofdepartment}{Dr. Supeno Mardi Susiki Nugroho, S.T., M.T.}

% identitas
\newcommand{\nrp}{0721 19 4000 0069}
\newcommand{\advisornip}{19750419 200212 1 003}
\newcommand{\coadvisornip}{19680601 199512 1 009}
\newcommand{\examineronenip}{XXXXXXXXXXXXXXXXX}
\newcommand{\examinertwonip}{XXXXXXXXXXXXXXXXX}
\newcommand{\examinerthreenip}{XXXXXXXXXXXXXXXXX}
\newcommand{\headofdepartmentnip}{19700313 199512 1 001}

% judul
\newcommand{\tatitle}{KONTROL PRESENTASI BERBASIS POSE TANGAN MENGGUNAKAN \emph{CONVOLUTIONAL NEURAL NETWORK} (CNN)}
\newcommand{\engtatitle}{HAND POSE BASED PRESENTATION CONTROL USING \emph{Convolutional Neural Network} (CNN)}

% tempat
\newcommand{\place}{Surabaya}

% jurusan
\newcommand{\studyprogram}{Teknik Komputer}
\newcommand{\engstudyprogram}{Computer Engineering}

% fakultas
\newcommand{\faculty}{Fakultas Teknologi Elektro dan Informatika Cerdas}
\newcommand{\engfaculty}{Faculty of Intelligent Electrical and Informatics Technology}

% singkatan fakultas
\newcommand{\facultyshort}{FTEIC}
\newcommand{\engfacultyshort}{FTEIC}

% departemen
\newcommand{\department}{Teknik Komputer}
\newcommand{\engdepartment}{Computer Engineering}

% kode mata kuliah
\newcommand{\coursecode}{EC224801}
