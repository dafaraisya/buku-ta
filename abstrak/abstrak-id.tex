\begin{center}
  \large\textbf{ABSTRAK}
\end{center}

\addcontentsline{toc}{chapter}{ABSTRAK}

\vspace{2ex}

\begingroup
% Menghilangkan padding
\setlength{\tabcolsep}{0pt}

\noindent
\begin{tabularx}{\textwidth}{l >{\centering}m{2em} X}
  Nama Mahasiswa    & : & \name{}         \\

  Judul Tugas Akhir & : & \tatitle{}      \\

  Pembimbing        & : & 1. \advisor{}   \\
                    &   & 2. \coadvisor{} \\
\end{tabularx}
\endgroup

% Ubah paragraf berikut dengan abstrak dari tugas akhir
Dalam melakukan presentasi terdapat beberapa pilihan cara kontrol yang tersedia saat ini, salah satunya menggunakan \emph{keyboard}. Namun, terdapat beberapa kekurangan dalam cara tersebut yaitu kontrol presentasi harus menekan tombol pada laptop secara langsung, sehingga tidak bisa dikendalikan dari jarak jauh dan dirasa kurang interaktif. Dalam penelitian ini, kontrol presentasi dilakukan menggunakan pose tangan sehingga tidak memerlukan kontak langsung lagi dengan perangkat yang digunakan untuk presentasi. Metode yang diterapkan sendiri menggunakan salah satu metode \emph{machine learning} dalam mengolah citra yaitu \emph{Convolutional Neural Network}. Berdasarkan pelaksanaan tugas akhir ini didapatkan hasil bahwa model dapat mendeteksi pose tangan dan terhubung dengan beberapa fungsi kontrol dalam aplikasi presentasi Microsoft PowerPoint menggunakan metode \emph{Convolutional Neural Network} dengan tingkat akurasi tertinggi sebesar 99.00\% pada jarak sekitar 40 cm dan kondisi cahaya minimal 40 lx. 

% Ubah kata-kata berikut dengan kata kunci dari tugas akhir
Kata Kunci: \emph{Pose Tangan, Presentasi, Convolutional Neural Network}.
