\begin{center}
  \large\textbf{ABSTRACT}
\end{center}

\addcontentsline{toc}{chapter}{ABSTRACT}

\vspace{2ex}

\begingroup
% Menghilangkan padding
\setlength{\tabcolsep}{0pt}

\noindent
\begin{tabularx}{\textwidth}{l >{\centering}m{3em} X}
  \emph{Name}     & : & \name{}         \\

  \emph{Title}    & : & \engtatitle{}   \\

  \emph{Advisors} & : & 1. \advisor{}   \\
                  &   & 2. \coadvisor{} \\
\end{tabularx}
\endgroup

% Ubah paragraf berikut dengan abstrak dari tugas akhir dalam Bahasa Inggris

\emph{When presenting, there are several control options currently available, one of which is using keyboard. However, there are several drawbacks in this method, namely the presentation control must press the buttons on the laptop directly, so it cannot be controlled remotely and feels less interactive. In this study, presentation control was carried out using hand poses so that it did not require direct contact with the device used for presentation. The self-implemented method uses one of the machine learning methods in processing images, namely Convolutional Neural Network. Based on the implementation of this final project, the results show that the model can detect hand poses and is connected to several control functions in the Microsoft PowerPoint presentation application using the Convolutional Neural Network method with an accuracy rate of 99\% in ideal conditions. The ideal conditions are at a distance of about 40 cm, light conditions around 40 lx, and using the author's hand in the test.}

% Ubah kata-kata berikut dengan kata kunci dari tugas akhir dalam Bahasa Inggris
\emph{Keywords}: \emph{Hand pose, Presentation, Convolutional Neural Network}.
