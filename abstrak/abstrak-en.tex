\begin{center}
  \large\textbf{ABSTRACT}
\end{center}

\addcontentsline{toc}{chapter}{ABSTRACT}

\vspace{2ex}

\begingroup
% Menghilangkan padding
\setlength{\tabcolsep}{0pt}

\noindent
\begin{tabularx}{\textwidth}{l >{\centering}m{3em} X}
  \emph{Name}     & : & \name{}         \\

  \emph{Title}    & : & \engtatitle{}   \\

  \emph{Advisors} & : & 1. \advisor{}   \\
                  &   & 2. \coadvisor{} \\
\end{tabularx}
\endgroup

% Ubah paragraf berikut dengan abstrak dari tugas akhir dalam Bahasa Inggris
\emph{When presenting, there are several control options that currently available, one of those is using mouse and keyboard. However, there are some drawbacks in this control method, like the presenter has to press the button on the laptop directly, so it cannot be controlled remotely and feels less interactive. In this study, presentation control will be carried out using hand pose so that it does not require direct contact with the device that used for the presentation. The method itself uses one of the machine learning methods for processing images, namely Convolutional Neural Network. Hopefully, this research can detect hand pose and connect it to several control functions in Microsoft PowerPoint presentation applications using the Convolutional Neural Network method.}

% Ubah kata-kata berikut dengan kata kunci dari tugas akhir dalam Bahasa Inggris
\emph{Keywords}: \emph{Hand pose, Presentation, Convolutional Neural Network}.
