\chapter{PENUTUP}
\label{chap:penutup}

% Ubah bagian-bagian berikut dengan isi dari penutup
Dalam bab ini berisi mengenai kesimpulan berdasarkan serangkaian rangkaian pengujian yang dipaparkan dalam Bab \ref{sec:hasilpengujian}, dan disesuaikan dengan tujuan yang berusaha dicapai dalam tugas akhir ini. Pada bab ini juga, terdapat saran-saran yang diberikan dalam rangka meningkatkan dan mengembangkan penelitian selanjutnya yang terkait dengan topik tugas akhir ini. Berikut ini adalah kesimpulan dan saran yang dapat diambil. 

\section{Kesimpulan}
\label{sec:kesimpulan}

Berdasarkan hasil pengujian yang sudah dipaparkan, dapat diambil beberapa kesimpulan sebagai berikut.

\begin{enumerate}[nolistsep]
  \item Klasifikasi citra dengan menggunakan metode Convolutional Neural Network (CNN) dapat digunakan untuk memprediksi citra pose tangan dengan tingkat akurasi mencapai 99.00\%. 
  \item Pada pengujian jarak dapat disimpulkan bahwa variasi pengambilan jarak dataset sangat mempengaruhi akurasi. Dimana, jarak penurunannya akurasinya bisa mencapai 30\%.
  \item Dari hasil pengujian variasi cahaya, disimpulkan bahwa model yang ada tidak mengalami penurunan yang signifikan baik ketika kondisi cahaya terang, redup, bahkan gelap. Penurunan yang terjadi masih dibawah 10\%.
  \item Berdasarkan pengujian \emph{frame rate} dengan menggunakan beberapa model CNN, dapat diambil kesimpulan bahwa kompleksitas dan besar ukuran model sangat mempengaruhi performa FPS-nya. 
  \item Dari hasil klasifikasi model yang didapatkan, dapat dihubungkan dengan berbagai fungsi navigasi ataupun penggunaan beberapa fitur kontrol dalam aplikasi Microsoft PowerPoint. 
  \item Kontrol presentasi menggunakan pose tangan dapat memberikan respon \emph{user experience} positif yang menarik dan mudah digunakan. 
\end{enumerate}   

\section{Saran}
\label{chap:saran}

Dalam rangka pengembangan penelitian kedepan terutama yang terkait dengan topik Convolution Neural Network, pose tangan, ataupun kontrol presentasi, dapat diusulkan beberapa saran yang disebutkan sebagai berikut.

\begin{enumerate}[nolistsep]
  \item Perlu ditambahkan jumlah dataset dengan berbagai variasi jarak pada saat pengambilan dataset. Sehingga diharapkan akurasi hasil klasifikasi tetap baik meskipun diambil dari jarak yang bervariasi.
  \item Menambahkan dataset dengan variasi ukuran tangan dari berbagai macam orang, agar akurasi yang dihasilkan lebih stabil meskipun diuji keresponden yang berbeda-beda.
\end{enumerate} 
